\documentclass{beamer}

\mode<presentation> {

%\usetheme{default}
%\usetheme{AnnArbor}
%\usetheme{Antibes}
%\usetheme{Bergen}
\usetheme{Berkeley}
%\usetheme{Berlin}
%\usetheme{Boadilla}
%\usetheme{CambridgeUS}
%\usetheme{Copenhagen}
%\usetheme{Darmstadt}
%\usetheme{Dresden}
%\usetheme{Frankfurt}
%\usetheme{Goettingen}
%\usetheme{Hannover}
%\usetheme{Ilmenau}
%\usetheme{JuanLesPins}
%\usetheme{Luebeck}
%\usetheme{Madrid}
%\usetheme{Malmoe}
%\usetheme{Marburg}
%\usetheme{Montpellier}
%\usetheme{PaloAlto}
%\usetheme{Pittsburgh}
%\usetheme{Rochester}
%\usetheme{Singapore}
%\usetheme{Szeged}
%\usetheme{Warsaw}

% As well as themes, the Beamer class has a number of color themes
% for any slide theme. Uncomment each of these in turn to see how it
% changes the colors of your current slide theme.

%\usecolortheme{albatross}
%\usecolortheme{beaver}
%\usecolortheme{beetle}
%\usecolortheme{crane}
%\usecolortheme{dolphin}
%\usecolortheme{dove}
%\usecolortheme{fly}
%\usecolortheme{lily}
%\usecolortheme{orchid}
%\usecolortheme{rose}
%\usecolortheme{seagull}
\usecolortheme{seahorse}
%\usecolortheme{whale}
%\usecolortheme{wolverine}

%\setbeamertemplate{footline} % To remove the footer line in all slides uncomment this line
%\setbeamertemplate{footline}[page number] % To replace the footer line in all slides with a simple slide count uncomment this line

%\setbeamertemplate{navigation symbols}{} % To remove the navigation symbols from the bottom of all slides uncomment this line
}

\usepackage{graphicx} % Allows including images
\usepackage{booktabs} % Allows the use of \toprule, \midrule and \bottomrule in tables

%----------------------------------------------------------------------------------------
%	TITLE PAGE
%----------------------------------------------------------------------------------------

\title[Card Payment Systems]{Card Payment Systems} % The short title appears at the bottom of every slide, the full title is only on the title page

\author{Akash Diwaker, Harsh Lathi	, Satyam Sachan} % Your name
\institute[IIT Bhilai] % Your institution as it will appear on the bottom of every slide, may be shorthand to save space
{
IIT Bhilai \\ % Your institution for the title page
\medskip
}
\date{\today} % Date, can be changed to a custom date

\begin{document}

	\begin{frame}
	\titlepage 
	\end{frame}
	
	\begin{frame}
		\frametitle{Overview} % Table of contents slide, comment this block out to remove it
		\tableofcontents % Throughout your presentation, if you choose to use \section{} and \subsection{} commands, these will automatically be printed on this slide as an overview of your presentation
	\end{frame}
	
	\section{Card Systems in Use}
	\begin{frame}
		\frametitle{Card Systems in Use}
		We shall focus on the three most commonly found card payment systems:\\
		
		\begin{itemize}
			\item Credit Cards
			\item Debit Cards
			\item Smart Cards
		\end{itemize}
		
	\end{frame}
	
	\section{Technologies Used}
	
	\begin{frame}
		\frametitle{Technologies Used}
		The most common technologies used to carry information on cards are as follows:
		\begin{itemize}
			\item Magnetic Stripes
			\item Contact-Based Chips
			\item Contactless Chips
		\end{itemize}
		The standard \texttt{ISO/IEC 7810} standardizes all the physical properties (such as the dimensions and material) of the cards.
	\end{frame}
	
	\subsection{Magnetic Stripe Cards}
	\begin{frame}
		\frametitle{Magnetic Stripe Cards}
		The implementation of the cards that use the magnetic stripe and corresponding standards used while creating said cards will be disccussed.\\ More particularly, parts 1 through 9 of the standard \texttt{ISO/IEC 7811} will be explained further within the context of the magnetic stripe cards.
	\end{frame}

	\begin{frame}
		\frametitle{Magnetic Stripe Cards}
		The most relevant standards are as follows:
		\begin{itemize}
			\item \texttt{ISO/IEC 7810} Physical characteristics of credit
			card size document
			\item \texttt{ISO/IEC 7811-1} Embossing
			\item \texttt{ISO/IEC 7811-2} Magnetic stripe - low coercivity
			\item \texttt{ISO/IEC 7811-3} Location of embossed characters
			\item \texttt{ISO/IEC 7811-4} Location of tracks 1 \& 2
			\item \texttt{ISO/IEC 7811-5} Location of track 3
			\item \texttt{ISO/IEC 7811-6} Magnetic stripe - high coercivity
			\item \texttt{ISO/IEC 7811-7} Magnetic stripe — High coercivity, high density
			\item \texttt{ISO/IEC 7811-8} Magnetic stripe -- Coercivity of 51.7 kA/m (650 Oersted)
			\item \texttt{ISO/IEC 7811-9} Tactile identifier mark
			\item \texttt{ISO/IEC 7813} Financial transaction cards
		\end{itemize}
	\end{frame}


	\subsection{Contact-Based Chips}
	\begin{frame}
		\frametitle{Contact-Based Chips}
		The implementation of the cards that use embedded chips and the corresponding standards used while manufacturing the cards will be discussed.\\
		The relevant parts of the standard \texttt{ISO/IEC 7816} will be discussed with respect to the contact-based chips.	
	\end{frame}

	\begin{frame}
		\frametitle{Contact-Based Chips}
		The most relevant standards are as follows:
		\begin{itemize}
			\item \texttt{ISO/IEC 7810 / 7816-1} Physical characteristics of credit
			card size document
			\item \texttt{ISO/IEC 7816-2} Dimensions and location of the contacts
			\item \texttt{ISO/IEC 7816-3} Electrical interface and transmission protocols
			\item \texttt{ISO/IEC 7816-4} Organization, security and commands for interchange
			\item \texttt{ISO/IEC 7816-8} Commands and mechanisms for security operations 
			\item \texttt{ISO/IEC 7816-9} Commands for card management
			\item \texttt{ISO/IEC 7816-15} Cryptographic information application
		\end{itemize}
	\end{frame}
	
	\subsection{Contactless Chips}
	\begin{frame}
		\frametitle{Contactless Chips}
		The implementation of the contactless (proximity and vicinity) cards and the corresponding standards used will be discussed.\\
		The defining standards \texttt{ISO/IEC 14443} (parts 1-4) and  \texttt{ISO/IEC 15693} will be further expanded upon. 
	\end{frame}
	
	\begin{frame}
		\frametitle{Contactless Chips}
		The most relevant standards are as follows:
		\begin{itemize}
			\item \texttt{ISO/IEC 7810} Physical characteristics of credit
			card size document
			\item \texttt{ISO/IEC 14443-2} RF Power and signal interface
			\item \texttt{ISO/IEC 14443-3} Initialization and anticollision
			\item \texttt{ISO/IEC 14443-4} Transmission protocol
			\item \texttt{ISO/IEC 15693} Contactless integrated circuit cards - Vicinity cards
		\end{itemize}
	\end{frame}

	
	\section{Transactions}
	
	\subsection{Protocols/Standards}
	\begin{frame}
		\frametitle{Transaction Protocols}
		The transaction protocols and standards specified for the above mentioned technologies and the respective PoS devices will be explained.\\
		The standard \texttt{ISO/IEC 8583} (the standard that financial organizations
		use to communicate and complete card transactions whether from an ATM, a PoS device, on the Internet or on a mobile network) will be discussed.
	\end{frame}
	
	
	\section{Security Systems}
	
	\begin{frame}
		\frametitle{Security Systems}
		The main focus of this section are the TLS/SSL (Transport Layer Security/Secure Socket Layer) protocols. These are used to secure exchanges in the transport layer of all parties in the transaction (the term 'transport layer' is from the TCP/IP layer model in networking where logical divisions are made to divide the tasks performed on the actual data before/when it is sent).
	\end{frame}
	
	
	
	
\end{document}